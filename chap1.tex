\chapter{Introduction}

Nuclear reactors have the highest energy density of any energy-producing technology currently available [cite].  This is due to their ultimate source of energy - the strong nuclear force, the strongest elementary force known in nature. The strong force binds atomic nuclei together and keeps matter stable.  When a heavy nucleus splits, as it does when a free neutron is absorbed by a fissile nucleus, it's daughter nuclei are more tightly bound than the parent, and the excess binding energy is released in various forms, most of which is deposited as heat near the split nucleus.  This heat can used to perform many tasks, but arguably the most useful task is to drive a conversion cycle used to convert a substantial amount of the heat into electricity.  Since the specific energy of nuclear fuel is around six orders of magnitude larger than chemical sources, the same fuel can be used in reactors for years, and the power cycle produces much less waste mass, even though its waste is very radioactive.  

As with anything very powerful, nuclear power must be handled with great responsibility.  The reactor's behavior must be very predictable and very well characterized in order for operators to prevent anything from going wrong and potentially releasing radioactivity into the open environment as well as to make sure the power plant is a reliable source of clean, affordable electric energy.  To ensure this required reliability and safety, very accurate simulations are needed in order to predict what will happen to the reactor if conditions within it change.  Since reactors are very expensive machines, accurate simulations are needed in the design phase as well; accurate enough to provide confidence to designers, regulators, and the public at large that a reactor will be safe before constructing a demonstration plant.  A popular way to conduct these simulations is to apply the Monte Carlo method to the neutron transport equations.  The Monte Carlo method requires few approximations to be made in the simulation model, and therefore can produce very physically-accurate results.  However, it is much more computationally expensive than other methods, and most often Monte Carlo simulations need to be run on supercomputers to produce results in a reasonable time for problems relevant to engineering.

General purpose graphical processing units (GPGPUs, referred to as GPUs henceforth) are an emerging computational tool, sporting very high memory bandwidth and computational throughput as well as lower power consumption per operation compared to standard CPUs.  Some applications can see upwards of a hundredfold speedup by running on GPUs.  They are touted as being ``massively parallel,'' home to thousands of computational ``cores,'' and capable of turning a desktop into a ``personal supercomputer.''  This makes them very attractive to use in extremely parallel, computationally-intensive simulations like Monte Carlo neutron transport where trillions, or more, of independent neutron lives are tracked.   

Some argue that the speedups gain are an illusion, and multicore CPUs are more than capable of similar performance if enough optimization is done. Some think that adding another programming paradigm is the wrong direction for computer science, and that it would be better if more resources were invested into existing technologies instead of spreading resources more thinly on new ones.  The an argument against both of these concerns is that CUDA (Compute Unified Device Architecture, NVIDIA's parallel computing platform) is fairly easy to program in and see substantial performance gains.  Maybe not 100x, but getting 10x is fairly easy.  This is due to the fact that GPUs were developed to be parallel from the beginning, unlike CPUs.  There is much less clutter and layering visible to the programmer, and CUDA is basically C with a few additional decorators, hardly a new language.   In addition, an empirical  bandwagon argument can be made for at least attempting to port codes to the GPU.  Many developers are porting and seeing reasonable speedups without any advanced training, more tat enough reason to at least attempt using CUDA. 

In this study, a simple question is proposed and answered: can the existing Monte Carlo algorithms be preserved or will they need to be rewritten in order to take full advantage of these new, powerful processor architectures?

\section{Why Monte Carlo?}

When applied to neutron transport, the central concept of the Monte Carlo method is to directly simulate what  microscopically happens to the neutrons in nature; every interaction they undergo, from birth to death.  Once a sufficiently large number of these ``histories'' are completed, sums and/or averages are taken over certain attributes to determine aggregate, macroscopic behavior.  Directly simulating what every individual neutron is doing is a rather brute-force way of simulating things since the macroscopic behavior is what matters in the end, but very few assumptions have to be made to conduct Monte Carlo simulations, making them one of the most accurate ways to simulate nuclear reactors.  Deterministic methods must discretize the spatial and energy domains, inevitably leading to unavoidable inaccuracies.  There is one huge drawback in using the Monte Carlo Method, however.  It is a statistical way of simulation and is subject to the laws of probability, especially the central limit theorem which stipulates slow convergence compared to deterministic approaches.  This is why any way of accelerating them is of interest and why GPUs are being studied in this work.  In other words, it is a simulation method that has a great need for acceleration. 

\section{Why GPUs?}

GPUs have gradually increased in computational power from the small, job-specific boards of the early 90s to the programmable powerhouses of today.  Compared to CPUs, they have a higher aggregate memory bandwidth, much higher floating-point operations per second (FLOPS), and lower energy consumption per FLOP.  Because one of the main obstacles in exascale computing is power consumption, many new supercomputing platforms are gaining much of their computational capacity by incorporating GPUs into their compute nodes.  In the Novermber 2013 Top500 list, there are 41 GPU-accelerated supercomputers, some of which gain 50\% of their computational capacity from GPU coprocessor cards \cite{top500}.  Supercomputers in the number two and six spots use them as well.  Since CPU optimized parallel algorithms are usually not directly portable to GPU architectures (or at least without losing substantial performance), transport codes may need to be rewritten in order to execute efficiently on GPUs.  Unless this is done, nuclear engineers cannot take full advantage of these new supercomputers for reactor simulations.

Table \ref{gpu_cpu_comp} shows a breakdown of features of both an AMD Opteron Magny-Cours CPU and an NVIDIA Tesla C2075 GPU.  At first glance, it may appear that the GPU completely outstrips the CPU.  It has higher single precision FLOPs (Floating Point OPeration), higher memory bandwidth, and higher concurrent thread capability.  These are all great features, and it seems that these cards would be perfect for running Monte Carlo neutron transport, especially due to the number of threads they can concurrently run.  The concurrent thread number is based on the width of the processor's SIMD lanes, however.  SIMD, or Single Instruction Multiple Data, is a way some processors run in order to lower then number of instructions needed per amount of computation done, which increases both power and computational efficiency.  SIMD requires the same instructions to be carried out over every element  in a concurrently-processed data vector, and Monte Carlo typically breaks this instruction regularity due to its conditional statements based on random numbers.  Therefore, if Monte Carlo algorithms are to be use on GPUs, they must be execute in a careful manner taking into account the limitations of the GPU.

\begin{table}[h]
\centering
\caption{A comparison of a NVIDIA GPU and an Opteron CPU \cite{cent}.}
\label{gpu_cpu_comp}
\begin{tabular}{| l | r | r |}
\hline
Processor & Intel Westmere-EP (i7) & NVIDIA Tesla C2070 (Fermi) \\
\hline
\hline
Processing Elements & 6 cores, 2 issue, & 16 cores, 2 issue, \\
& 4-way SIMD &  16-way SIMD  \\
\hline
Frequency & 3.46GHz &  1.54 Ghz \\
\hline
Resident Strands / Threads (max) & 48 & 24576 \\
\hline
SP GFLOP/s & 166 & 1577 \\
\hline
Mem. Bandwidth &  32 GB/s & 192 GB/s \\
\hline
Global Latency & 200-800 clocks & ~50 clocks \\
\hline
FLOPs / byte & 12.5  & 16.1 \\
\hline
Register Fils & 6kB (?) & 2MB \\
\hline
Local Storage / L1 Cache & 192 kB & 1024 kB \\
\hline
L2 Cache & 1536 kB & 0.75 MB \\
\hline
L3 Cache & 12 MB & - \\
\hline
\end{tabular}
\end{table}

Another considerably undesirable feature of the GPU is that they have very high global memory latency compared to a CPU.  As Table \ref{gpu_cpu_comp} shows, the Fermi card's global memory latency is about an order of magnitude higher than the Operton's \cite{cpu_latency,cuda}.  Since data is accessed in a very random way in Monte Carlo simulations, access regularity can again be broken, and the global access latency could slow a simulation down significantly.  GPUs try to eliminate the effect of their large global latency by pipelining memory access as well as relaying on access regularity.  Pipelining means threads that have their data can execute as other threads are waiting for their data to load.  If many requests are known, the data can be continually loaded as threads start to execute their jobs.  The hope is that the jobs take longer than the memory loads, and eventually all data arrives, and the later threads appear to have zero latency for their memory access.  This is why it is important for GPUs to have such a large number of concurrent threads.  It allows them to pipeline data access and minimize the impact of memory latency.

Another notable feature is that the GPU has greater FLOPs /byte of memory bandwidth ratio.  This implies that GPUs could be used to turn a compute-bound problem into a bandwidth-bound problem.  This may seem like a deficit, but the GPU does, after all, have a higher maximum memory bandwidth, so even though a problem is memory-bound on a GPU, it may still perform better than on a CPU.

Table \ref{gpu_money} shows a comparison of an AMD Opteron Magny-Cours CPU and an NVIDIA Tesla C2075 GPU \cite{cpu_latency,cuda}.  This prices shown are rounded values from purchases made by the UC Berkeley Department Nuclear Engineering.  The CPU price is for a complete server, basically CPUs and RAM.  The disk and  other components are a secondary cost.  The price shown for the GPUs are for the cards only since they can live in a very cheap host server with very little RAM and budget CPUs.  The main benefit from using GPUs is not the capital price per FLOP, but rather the substantially lower electric cost per FLOP.  Also, assuming that a GPU application performs 25 times faster than a single CPU code and the CPU code scales linearly, the capital price per Monte Carlo ``history power,'' or histories run per second, is about 3 times that of CPUs.  Again, this assumes 25 times speedup over a CPU code.  

\begin{table}[h]
\centering
\caption{Breakdown of the cost benefits of GPUs}
\label{gpu_money}
\begin{tabular}{| l | r | r |}
\hline
Processor & 4x Opteron 12-core  & 3x NVIDIA TESLA C2075 \\
  & @ 2.1 GHz &  (cards only)  \\
\hline
\hline
Price & \$10,000 & \$6,000 \\
\hline
Max.TeraFLOP & 0.4 & 3.1 \\
\hline
Price / GigaFLOP & \$25 & \$1.94 \\
\hline
Price / History Power (10$^3$ h / s) & \$1.00 & \$0.38 \\
(assuming 25x GPU speedup) & & \\
\hline
Thermal Power & 225W & 130W \\
\hline
Yearly electricity cost & \$503.70  & \$95.37 \\
per TeraFLOP (\$.05 / kWh)   & & \\
\hline
\end{tabular}
\end{table}

\section{Added Value}

Producing this program, which has been named WARP, will be the first step in hopefully creating a full featured reactor simulation program that runs on GPU accelerator cards.  The added value in doing this comes from the fact that many supercomputers are gaining power from GPUs and in order for nuclear engineers to take advantage of this new computing power, which they always need more of, a new code must be written to run on them.  Even though GPU computing is still in its very early stages, developing WARP hedges risk for the simulation community against their computational tools becoming under powered or even obsolete.  Conversely, it also exposes a substantial amount of  computing power that couldn't otherwise be used on personal computers, workstations, and laptops.  Targeting GPUs could enable smaller computers to take on substantial reactor simulation tasks, and potentially reduce design iterations for people without access to supercomputers.  Other than being a nod to NVIDIA's terminology, if WARP were an acronym, it would stand for ``weaving all the random particles,'' with the words ``weaving all'' referring to the lockstep way in which ``all the random particles'', i.e. the neutrons, are sorted into coherent bundles and transported.

\section{Objective}

The first goal of this work is first to produce a program (WARP) that runs accurate continuous energy neutron transport simulations on the GPU in general 3D geometries using standard nuclear data files.  This program will be able to run in both fixed source and criticality source modes.  It will also be able to produce neutron spectra.  The second goal is to make these simulations as fast as possible.

\section{Outline}

1 paragraph summaries of each chapter.  Will do once they are written!
