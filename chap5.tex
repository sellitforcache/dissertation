\chapter{Conclusions and Future Work}
\label{chap:conclusions}

important to keep saturated (criticality is a global quantity, using global algorithm, no state continuity between launches so must be stored in global memory, therefore very important to launch as many threads as possible to keep things pipelined and the device saturated), each history requires XXX bytes of data on top of the data required by optix, means maximum dataset size is about XXX histories on a k20 card with 5GB (assuming optix and tally memory is negligible)

low number not not saturate the gpu, not enough payload for the overhead , pipelining

replace geom, parallel routines with handwritten, specific ones, if geom could fit in shared memory, there would be great performance increase
- use a SM-based algorithm instead of global.  history info could be stored in shared, but would need to rendezvous.
- newer libraries, CUB, optix prime, rayforce?
- legendre instead of tabular angular?
- better geom performance with woodcock maybe
- material processing schemes
- dynamic parallelism can be implemented and kernel launch overhead minimized.  
- change fixed source pop to be more like criticaility, pop into next generation instead of this generation?
-----  might be the same as using a stack-pop and task based with syncthreads?  The only routines used in the transport loop are the geometry and the sort.   pop would replace the sort?  and pass off reactions if total coherence is wanted.  still need to communicate the fissions?  this could be eliminated by a query-able analytic representation of a converged fission source.  FFT, something else.  Might not be fast as simply using the previous cycle points but would scale well?  Previous points could be used for refinement...  might be a neat consideration?