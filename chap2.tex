\chapter{Background}


%%%%%%%%%%%%%%%%%%%%%%%%%%%%%%

\section{Reactor Analysis}

What its all about

\subsection{History}

The Monte Carlo method is not a new way to simulate nuclear reactors (or any particle transport problem for that matter) and in it's modern form dates back to 1940 when XXX invented the approach while working on the Manhattan Project[cite].  During this time, Henry Metropolis and John Von Neumann...[cite]  Although this may be disputed by some people since is was rumored that Enrico Fermi was basically doing small scale simulations of this kind in his head trying to get the Chicago Pile critical [cite].

\subsection{Nuclear Interactions}

\subsection{Neutron Transport}

NTE

\subsection{Discrete Methods}

Diffusion, SN

drawbacks, limitations

\subsection{Monte Carlo}

equations and derivations and such

scaling

advantages

drawbacks and limitations

\subsection{Nuclear Data}

it's all about the data

%%%%%%%%%%%%%%%%%%%%%%%%%%%%%%

\section{GPUs}

Part of the reason GPUs are able to perform efficiently is do to their reliance on single instruction, multiple data (SIMD) execution.  This execution method uses the same instructions carried out over multiple pieces of data at the same time.  This reduces the amount of power used in control and therefore more math can be done per watt [citation definitely needed].  The GPU programming model abstracts SIMD execution by using threads, which can be thought of in the traditional sense . There are some tradeoffs, however, including limited on-board memory (currently 12 GB), limited cache and control space, and requiring data parallelism for full utilization.

Since all threads in a GPU thread block must execute the same instructions, having conditional statements based on random numbers can cause threads to be serialized, leading to resource under-utilization.  For example, in particle transport where a particle history is assigned to each thread, thread divergence mainly arises through particles undergoing different reactions based on the results of cross-section sampling.  Also, GPUs have even higher global memory latency than CPUs, up to 800 clock cycles on older cards, compared to about 50 cycles on a typical CPU.  It is also important to remember than GPUs are clocked much slower than CPUs, so the memory latency performance issue is even greater [5, 7]. 

another brief introduction.  

advantages

pitfalls, weaknesses

\subsection{Supercomputing}

how and why they came to be used for supercomputing.

\subsection{Architecture}

\subsection{Memory}

speed, levels, etc.  limitations 

%%%%%%%%%%%%%%%%%%%%%%%%%%%%%%

\section{Previous Works}

talk about the work that other people did

\section{Preliminary Studies}

all my prelim work to motivate this project


%%%%%%%%%%%%%%%%%%%%%%%%%%%%%%

\section{Scope in Detail}


